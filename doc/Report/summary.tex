\section{Conclusion and Outlook}
\subsection{Conclusion} \label{summary}
As was seen in section \ref{results}, the results suggest the amount of cyclists in traffic does not have much affect on the average speed, density or flow of road users. The slight decrease in average speed is negligable for the short distances, which are travelled inside of a city like Zurich. This would make a difference of mere seconds on a trip. When looking at the density or flow in section \ref{density} and \ref{flow}, the amount of cyclists  does not suggest a meaningful relief from congestions, nor is the average speed increased. All in all, our results show no meaningful influence of an increase in cyclists on traffic. Nevertheless, there are other reasons to move away from motorised individual transport to bicycling or usage of public transport, like a quieter and more peaceful city and less environmental pollution. These goals are also worth pursuing, especially since an increase in cyclists does not suggest a worsening of the traffic situation in our simulation.\\
However, as seen and described in section \ref{avg-speed}, there are some slight trends. The differences between the different values are small and negligible. Due to the abstractions mentioned in section \ref{abstractions}, not accounting for some behavior of agents and other influential factors on traffic in our model, combined with only slight trends in the result, this does not allow for the results to be transferred to reality one-to-one.\\

\subsection{Outlook}
There are many things a future model could expand upon. An immediate and natural expansion would be to include other users of the road network, such as public transport and pedestrians. Furthermore, bicycling is heavily dependant on environmental factors, such as weather and seasons. While the weather is somewhat difficult to model, the seasons could be an interesting way to account for a coarser trend. \\
Other than that, further improvements could be made by expanding the existing model with more detail. This could include more realistic routes for the agents, by regarding existing traffic when calculating routes, instead of precomputing all the shortest routes solely based on the length of the roads. This would be possible by calculating the routes iteratively and regard the current traffic situation when fixing the edge weights. By choosing an appropriate time step for the iterations, this would allow for a balance between computational resources needed and the accuracy of the model. More complex behavior of the agents is also a point, where current abstractions could be eliminated. This could include overtaking on opposite lanes if free and not only, if the immediate left lane is free. Furthermore the driving styles could be randomized a bit further and not only be dependent on the given attributes.\\
Another possibility to make the model more accurate, is to add more realistic travel data. Although hard to get, meaningful travel data, like trends in the start and destination of inner city travel or the amount of agents over the course of a day could drastically increase the significance of results.  Finally, one could always expand the model simply by using a more precise road network. 
