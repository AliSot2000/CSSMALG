\section{Abstract}
Due to the pandemic and also the city of Zurich pushing for more environmentally friendly transportation, bicycle usage has seen a growth in recent years. This report entails a description of an Agent Based Model, which aims to simulate the traffic of Zurich under the influence of a varying amount of cyclists on the road. The model consists of an environment, given by the road network accessible by cars and bicycles, depicted by agents. It found that higher percentages of bicycles with a consistent absolute amount of agents does not result in any significant changes. Part of this could be caused due to certain abstractions made, which are mentioned in section \ref{abstractions}.\\
Nevertheless, the absence of significant changes in traffic flow, density or average speed, as described in section \ref{results}, is not necessarily something bad. This also allows the conclusion that an increase in cyclists does not cause an increase in congestions or travel time of commuters. Therefore it is not a solution to traffic jams, but does also not cause more. This allows an increase of cyclists in traffic, if aspired for other reasons, without risking a more congested city. These reasons could include a safer and quieter city, as well as less environmental pollution.