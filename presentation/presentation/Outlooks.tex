
\begin{frame}[containsverbatim, fragile]{Abstractions}
\begin{itemize}
    \item<2->[-] Model only considers cars and bikes, no public transport or other vehicles
    \item<3->[-] Couldn't find proper travel data
    \begin{itemize}
        \item<4->[-] Start and end points are randomized
        \item<5->[-] Amount of agents is evenly distributed over the simulation
    \end{itemize}
    \item<6->[-] Cutoff point for streets was chosen slightly higher than real world counterpart
    \item<7->[-] Model doesn't incorporate driving styles, lawfulness
    \item<8->[-] No consideration for extraordinary events: Accidents, Road Construction
\end{itemize}
\end{frame}

\begin{frame}[containsverbatim, fragile]{Conclusion}
\begin{itemize}
    \item<2->[-] no big effect on average speed, density or flow
    \item<3->[-] no significant change in density
    \begin{itemize}
        \item<4->[-] no congestion relief
        \item<5->[-] no negative impact either
    \end{itemize}
    \item<6->[-] other reasons for changing to bicycles
    \begin{itemize}
        \item<7->[-] less environmental pollution
        \item<8->[-] quieter city
    \end{itemize}
\end{itemize}
\end{frame}

\begin{frame}[containsverbatim, fragile]{Further Extensions}
In a continuation of this project, things that could be of interest to study could be:
\begin{itemize}
    \item<2->[-] Including public transport, pedestrians, etc.
    \item<3->[-] Seasons, as bicycle usage is weather dependant
    \item<4->[-] More realistic routes and agent amount
    \item<5->[-] Larger road network
    \item<6->[-] Higher complexity for agent behavior, overtaking on opposite roads etc.
    \item<7->[-] Calculating the associated cost difference
\end{itemize}
\end{frame}
